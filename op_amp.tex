% PAGES 4-5
\begin{table}[h]
	\begin{center}
		\begin{tabular}{|CCC|}
			\hline
			\multicolumn{3}{|C|}{\textbf{Ζεύγος κοινού συλλέκτη - κοινής βάσης:}}                                                                                                                                                                                                                                                                                                          \\
			\frac{V_o}{V_{sig}}=\frac{1}{2}\(\frac{R_{in}}{R_{in}+R_{sig}}\)\(g_mR_L\)                           & R_{in}=2r_\pi                                                                                                                                                 &                                                                                                         \\
			f_{P1}=\frac{1}{2\pi\(\frac{C_\pi}{2}+C_\mu\)(R_{sig}\parallel 2r_\pi)}                              & f_{P2}=\frac{1}{2\pi C_\mu R_L}                                                                                                                               & f_H\cong\displaystyle{\sfrac{1}{\sqrt{\frac{1}{f^2_{P1}}+\frac{1}{f^2_{P2}}}}}                          \\
			\multicolumn{3}{|C|}{\textbf{Ζεύγος κοινής πηγής - κοινής πύλης (κασκοδική συνδεσμολογία):}}                                                                                                                                                                                                                                                                                   \\
			R_{out}=r_{o2}+\left[1+\(g_{m2}+g_{mb2}\)r_{o2}\right]r_{o1}                                         & A_v=-A_0^2\frac{R_L}{R_L+A_0r_0}                                                                                                                              & f_H\cong\frac{1}{2\pi\tau_H}                                                                            \\
			R_{gd1}=\(1+g_{m1}R_{d1}\)R_{sig}+R_{d1}                                                             & \multicolumn{2}{C|}{\tau_H=R_{sig}\left[C_{gs1}+C_{gd1}\(1+g_{m1}R_{d1}\)\right]+R_{d1}\(C_{gd1}+C_{db1}+C_{gs2}\)+\(R_L\parallel R_{out}\)\(C_L+C_{gd2}\)}                                                                                                             \\
			\multicolumn{3}{|C|}{\textbf{Ζεύγος κοινού εκπομπού - κοινής βάσης (κασκοδική συνδεσμολογία):}}                                                                                                                                                                                                                                                                                \\
			A_M=-\frac{r_\pi}{r_\pi+r_x+R_{sig}}g_m\(\beta r_0\parallel R_L\)                                    & R_{c1}=r_{01}\parallel\left[r_{e2}\(\frac{r_{o2}+R_L}{r_{o2}+\displaystyle{\sfrac{R_L}{\(\beta_2+1\)}}}\)\right]                                              &                                                                                                         \\
			R_{sig}^\prime=r_{\pi 1}\parallel\(r_{x1}+R_{sig}\)                                                  & \multicolumn{2}{C|}{\tau_H=C_{\pi 1}R_{\pi 1}+C_{\mu 1}R_{\mu 1}+\(C_{cs1}+C_{\pi 2}\)R_{c1}+\(C_L+C_{cs2}+C_{\mu 2}\)\(R_L\parallel R_{out}\)}                                                                                                                         \\
			R_{\mu 1}=R_{sig}^\prime\(1+g_{m1}R_{c1}\)+R_{c1}                                                    & f_H\simeq\frac{1}{2\pi\tau_H}                                                                                                                                 &                                                                                                         \\
			\multicolumn{3}{|C|}{\textbf{Τελεστικός ενισχυτής MOS δύο βαθμίδων:}}                                                                                                                                                                                                                                                                                                          \\
			A_v=-g_{m1}\(r_{ds2}\parallel r_{ds4}\)                                                              & \multicolumn{2}{C|}{g_{m1}=\sqrt{\displaystyle{2\mu_p C_{ox}\(\frac{W}{L}\)_1I_{D1}}}=\sqrt{\displaystyle{2\mu_p C_{ox}\(\frac{W}{L}\)_1\frac{I_{bias}}{2}}}}                                                                                                           \\
			\multicolumn{3}{|C|}{R_B=\frac{2}{\sqrt{\displaystyle{2\mu_nC_{ox}\(\sfrac{W}{L}\)_{12}I_B}}}\(\sqrt{\displaystyle{\frac{\(\sfrac{W}{L}\)_{12}}{\(\sfrac{W}{L}\)_{13}}}}-1\)}                                                                                                                                                                                                  \\
			g_{m12}=\frac{2}{R_B}\(\sqrt{\displaystyle{\frac{\(\sfrac{W}{L}\)_{12}}{\(\sfrac{W}{L}\)_{13}}}}-1\) & g_{mi}=g_{m12}\sqrt{\displaystyle{\frac{I_{Di}\(\sfrac{W}{L}\)_i}{I_B\(\sfrac{W}{L}\)_{12}}}}                                                                 & g_{mi}=g_{m12}\sqrt{\displaystyle{\frac{\mu_pI_{Di}\(\sfrac{W}{L}\)_i}{\mu_nI_B\(\sfrac{W}{L}\)_{12}}}} \\
			C_1=C_{gd4}+C_{db4}+C_{gd2}+C_{gs6}                                                                  & C_2=C_{db6}+C_{db7}+C_{gd7}+C_L                                                                                                                               &                                                                                                         \\
			\omega_Z=\frac{G_{m2}}{C_C}                                                                          & \multicolumn{2}{C|}{\omega_{P1}=\frac{1}{C_1R_1+C_2R_2+C_C\(G_{m2}R_2R_1+R_1+R_2\)}\cong\frac{1}{R_1C_CG_{m2}R_2}}                                                                                                                                                      \\
			\omega_{P2}=\frac{G_{m2}C_C}{C_1C_2+C_C\(C_1+C_2\)}                                                  & \omega_t=\(G_{m1}R_1G_{m2}R_2\)\omega_{P1}                                                                                                                    & \text{Επιλογή }C_C\text{ ώστε }\omega_t<\omega_Z<\omega_{P2}                                            \\
			\mathrm{PSRR}=g_{mN}\(r_{oP}\parallel r_{oN}\)                                                       & \mathrm{SR}\equiv\left.\der{V_{out}}{t}\right\vert_{\max}=\frac{I_{SS}}{C_L}                                                                                  &                                                                                                         \\
			\multicolumn{3}{|C|}{\mathrm{GB}=A_v(0)\cdot|p_1|=\(g_{m1}g_{m2}R_IR_{II}\)\cdot\(\frac{1}{g_{m2}R_IR_{II}C_C}\)=\frac{g_{m1}}{C_C}}                                                                                                                                                                                                                                           \\
			\multicolumn{3}{|C|}{\mathrm{PM}=\mathrm{Arg}\left[AB\right]=\pm180{\degree}-\arctan{\(\frac{\omega}{|p_1|}\)}-\arctan{\(\frac{\omega}{|p_2|}\)}-\arctan{\(\frac{\omega}{Z}\)}}                                                                                                                                                                                                \\
			\multicolumn{3}{|l|}{Για περιθώριο φάσης (PM) $45\degree$ και $Z\geqslant 10\cdot\mathrm{GB}$: $|p_2|\geqslant 1.22\cdot\mathrm{GB}$}                                                                                                                                                                                                                                          \\
			\multicolumn{3}{|l|}{Για περιθώριο φάσης (PM) $60\degree$ και $Z\geqslant 10\cdot\mathrm{GB}$: $|p_2|\geqslant 2.22\cdot\mathrm{GB}$}                                                                                                                                                                                                                                          \\
			\multicolumn{3}{|C|}{Z=\frac{1}{C_C\(\displaystyle{\frac{1}{g_{m2}}}-R_Z\)}}                                                                                                                                                                                                                                                                                                   \\[18pt]
			\hline
		\end{tabular}
	\end{center}
\end{table}

\begin{table}[h]
	\begin{center}
		\begin{tabular}{|CCC|}
			\hline
			\multicolumn{3}{|c|}{\textbf{Τελεστικός ενισχυτής με είσοδο n-MOS}}                                                                                                                                                                                                          \\
			\max{\(V_{in}\)}=V_{DD}-\sqrt{\displaystyle{\frac{I_5}{\beta_3}}}-\max{\(|V_{T03}|\)}+\min{\(V_{T01}\)} & \min{\(V_{in}\)}=V_{SS}+\sqrt{\displaystyle{\frac{I_5}{\beta_1}}}+\max{\(V_{T01}\)}+V_{DS5_{sat}}       & V_{DS_{sat}}=\sqrt{\displaystyle{\frac{2I_{DS}}{\beta}}} \\[5pt]
			\hline
			\multicolumn{3}{|c|}{\textbf{Τελεστικός ενισχυτής με είσοδο p-MOS}}                                                                                                                                                                                                          \\
			\max{\(V_{in}\)}=V_{DD}-\sqrt{\displaystyle{\frac{I_5}{\beta_1}}}-V_{DS5_{sat}}-\max{\(|V_{T01}|\)}     & \min{\(V_{in}\)}=V_{SS}+\sqrt{\displaystyle{\frac{I_5}{\beta_3}}}+\max{\(V_{T03}\)}-\min{\(|V_{T01}|\)} &                                                          \\[5pt]
			\hline
		\end{tabular}
	\end{center}
\end{table}

\begin{table}[h]
	\begin{center}
		\begin{tabular}{|CCCC|}
			\multicolumn{4}{c}{\textbf{Σχεδίαση τελεστικού ενισχυτή με είσοδο n-MOS}}                                                                                                                                                                                                                                                                                                                                                                                                                                 \\
			\hline
			\mathrm{SR}=\frac{I_5}{C_C}                                                                                                                           & \multicolumn{2}{C}{A_{v1}=\frac{-g_{m1}}{g_{ds2}+g_{ds4}}=\frac{-2g_{m1}}{I_5\cdot\(\lambda_2+\lambda_4\)}} & A_{v2}=\frac{-g_{m6}}{g_{ds6}+g_{ds7}}=\frac{-g_{m6}}{I_6\cdot\(\lambda_6+\lambda_7\)}                                                                                                                                              \\
			\mathrm{GB}=\frac{g_{m1}}{C_C}                                                                                                                        & p_2=\frac{-g_{m6}}{C_L}                                                                                     & Z=\frac{g_{m6}}{C_C}                                                                                                                          & \beta=k^\prime\frac{W}{L}\cong\mu_0C_{ox}\frac{W}{L}\;\(\unit{\ampere\per\volt^2}\) \\[5pt]
			C_C>0.22C_L                                                                                                                                           & I_5=\mathrm{SR}\cdot C_C                                                                                    & \multicolumn{2}{C|}{S_3=\(\frac{W}{L}\)_3=\frac{I_5}{k_3^\prime\left[V_{DD}-\max{\(V_{in}\)}-\max{\(|V_{T03}|\)}+\min{\(V_{T01}\)}\right]^2}}                                                                                       \\[7pt]
			\frac{g_{m3}}{2C_{gs3}}>10\cdot\mathrm{GB}                                                                                                            & \multicolumn{3}{C|}{g_{m1}=\mathrm{GB}\cdot C_C\Longrightarrow S_1=S_2=\frac{g_{m2}^2}{k_2^\prime I_5}}                                                                                                                                                                                                                                           \\
			\multicolumn{3}{|C}{V_{DS5_{sat}}=\min{\(V_{in}\)}-V_{SS}-\sqrt{\displaystyle{\frac{I_5}{\beta_1}}}-\max{\(V_{T01}\)}\geqslant 100\unit{\milli\volt}} & S_5=\frac{2I_5}{k_5^\prime V_{DS5_{sat}}^2}                                                                                                                                                                                                                                                                                                       \\
			g_{m6}=2.2g_{m2}\(\sfrac{C_L}{C_C}\)                                                                                                                  & S_6=S_4\frac{g_{m6}}{g_{m4}}                                                                                & I_6=\frac{g_{m6}^2}{2k_6^\prime S_6}                                                                                                          &                                                                                     \\
			S_6=\frac{g_{m6}}{k_6^\prime V_{DS6_{sat}}}                                                                                                           & \multicolumn{2}{C}{V_{DS6}=\min{\(V_{DS6}\)}=V_{DS6_{sat}}=V_{DD}-\max{\(V_{out}\)}}                        & S_7=S_5\frac{I_6}{I_5}                                                                                                                                                                                                              \\
			\multicolumn{2}{|C}{A_v=\frac{2g_{m2}g_{m6}}{I_5\(\lambda_2+\lambda_4\)I_6\(\lambda_6+\lambda_7\)}}                                                   & \multicolumn{2}{C|}{P_{diss}=\(I_5+I_6\)\cdot\(V_{DD}+|V_{SS}|\)}                                                                                                                                                                                                                                                                                 \\[7pt]
			\hline
		\end{tabular}
	\end{center}
\end{table}