% PAGES 4-5
\begin{table}[h]
	\begin{center}
		\begin{tabular}{|CCC|}
			\multicolumn{3}{c}{\textbf{Πολυβάθμιοι ενισχυτές}}                                                                                                                                                                                                                                                                                                                               \\
			\hline
			\multicolumn{3}{|C|}{A_v=\frac{e_{\mathrm{out}}}{V_S}=\prod_{i=1}^{n}{A_{i}}=\frac{R_{in,1}}{R_{in,1}+R_S}\cdot\frac{R_{in,2}}{R_{in,2}+R_{out,1}}\cdot\ldots\cdot\frac{R_L}{R_L+R_{out,n}}}                                                                                                                                                                                              \\
			\text{Με $n$ όμοιες βαθμίδες: } \omega_{nL}=\frac{\omega_0}{\sqrt{\displaystyle{2^{\sfrac{1}{n}-1}}}}. & \multicolumn{2}{C|}{\text{Με $n$ βαθμίδες με $f_{Lj},\;j=1,\ldots,n$: } f_{nL}\approx1.1\sqrt{\sum_{j=1}^{n}{f_{Lj}^2}}}                                                                                                                                                \\
			\text{Με $n$ όμοιες βαθμίδες: } \omega_{nH}=\omega_0\sqrt{\displaystyle{2^{\sfrac{1}{n}-1}}}.          & \multicolumn{2}{C|}{\text{Με $n$ βαθμίδες με $f_{Hj},\;j=1,\ldots,n$: } f_{nH}\approx\(1.1\sqrt{\displaystyle{\sum_{j=1}^{n}{f_{Hj}^2}}}\)^{-1}}                                                                                                                        \\
			\multicolumn{3}{|C|}{\textbf{Ζεύγος κοινού συλλέκτη - κοινής βάσης:}}                                                                                                                                                                                                                                                                                                            \\
			\frac{V_o}{V_{\mathrm{sig}}}=\frac{1}{2}\(\frac{R_{\mathrm{in}}}{R_{\mathrm{in}}+R_{\mathrm{sig}}}\)\(g_mR_L\)                             & R_{\mathrm{in}}=2r_\pi                                                                                                                                                 &                                                                                                         \\
			f_{P1}=\frac{1}{2\pi\(\frac{C_\pi}{2}+C_\mu\)(R_{\mathrm{sig}}\parallel 2r_\pi)}                                & f_{P2}=\frac{1}{2\pi C_\mu R_L}                                                                                                                               & f_H\cong\frac{1}{\sqrt{\displaystyle{\frac{1}{f^2_{P1}}+\frac{1}{f^2_{P2}}}}}                          \\
			\multicolumn{3}{|C|}{\textbf{Ζεύγος κοινής πηγής - κοινής πύλης (κασκοδική συνδεσμολογία):}}                                                                                                                                                                                                                                                                                     \\
			R_{\mathrm{out}}=r_{o2}+\left[1+\(g_{m2}+g_{mb2}\)r_{o2}\right]r_{o1}                                           & A_v=-A_0^2\frac{R_L}{R_L+A_0r_0}                                                                                                                              & f_H\cong\frac{1}{2\pi\tau_H}                                                                            \\
			R_{gd1}=\(1+g_{m1}R_{d1}\)R_{\mathrm{sig}}+R_{d1}                                                               & \multicolumn{2}{C|}{\tau_H=R_{\mathrm{sig}}\left[C_{gs1}+C_{gd1}\(1+g_{m1}R_{d1}\)\right]+R_{d1}\(C_{gd1}+C_{db1}+C_{gs2}\)+\(R_L\parallel R_{\mathrm{out}}\)\(C_L+C_{gd2}\)}                                                                                                             \\
			\multicolumn{3}{|C|}{\textbf{Ζεύγος κοινού εκπομπού - κοινής βάσης (κασκοδική συνδεσμολογία):}}                                                                                                                                                                                                                                                                                  \\
			A_M=-\frac{r_\pi}{r_\pi+r_x+R_{\mathrm{sig}}}g_m\(\beta r_0\parallel R_L\)                                      & R_{c1}=r_{01}\parallel\left[r_{e2}\(\frac{r_{o2}+R_L}{r_{o2}+\displaystyle{\sfrac{R_L}{\(\beta_2+1\)}}}\)\right]                                              &                                                                                                         \\
			R_{\mathrm{sig}}^\prime=r_{\pi 1}\parallel\(r_{x1}+R_{\mathrm{sig}}\)                                                    & \multicolumn{2}{C|}{\tau_H=C_{\pi 1}R_{\pi 1}+C_{\mu 1}R_{\mu 1}+\(C_{cs1}+C_{\pi 2}\)R_{c1}+\(C_L+C_{cs2}+C_{\mu 2}\)\(R_L\parallel R_{\mathrm{out}}\)}                                                                                                                         \\
			R_{\mu 1}=R_{\mathrm{sig}}^\prime\(1+g_{m1}R_{c1}\)+R_{c1}                                                      & f_H\simeq\frac{1}{2\pi\tau_H}                                                                                                                                 &                                                                                                         \\
			\multicolumn{3}{|C|}{\textbf{Τελεστικός ενισχυτής MOS δύο βαθμίδων:}}                                                                                                                                                                                                                                                                                                            \\
			A_v=-g_{m1}\(r_{ds2}\parallel r_{ds4}\)                                                                & \multicolumn{2}{C|}{g_{m1}=\sqrt{\displaystyle{2\mu_p C_{ox}\(\frac{W}{L}\)_1I_{D1}}}=\sqrt{\displaystyle{2\mu_p C_{ox}\(\frac{W}{L}\)_1\frac{I_{bias}}{2}}}}                                                                                                           \\
			\multicolumn{3}{|C|}{R_B=\frac{2}{\sqrt{\displaystyle{2\mu_nC_{ox}\(\sfrac{W}{L}\)_{12}I_B}}}\(\sqrt{\displaystyle{\frac{\(\sfrac{W}{L}\)_{12}}{\(\sfrac{W}{L}\)_{13}}}}-1\)}                                                                                                                                                                                                    \\
			g_{m12}=\frac{2}{R_B}\(\sqrt{\displaystyle{\frac{\(\sfrac{W}{L}\)_{12}}{\(\sfrac{W}{L}\)_{13}}}}-1\)   & g_{mi}=g_{m12}\sqrt{\displaystyle{\frac{I_{Di}\(\sfrac{W}{L}\)_i}{I_B\(\sfrac{W}{L}\)_{12}}}}                                                                 & g_{mi}=g_{m12}\sqrt{\displaystyle{\frac{\mu_pI_{Di}\(\sfrac{W}{L}\)_i}{\mu_nI_B\(\sfrac{W}{L}\)_{12}}}} \\
			C_1=C_{gd4}+C_{db4}+C_{gd2}+C_{gs6}                                                                    & C_2=C_{db6}+C_{db7}+C_{gd7}+C_L                                                                                                                               &                                                                                                         \\
			\omega_Z=\frac{G_{m2}}{C_C}                                                                            & \multicolumn{2}{C|}{\omega_{P1}=\frac{1}{C_1R_1+C_2R_2+C_C\(G_{m2}R_2R_1+R_1+R_2\)}\cong\frac{1}{R_1C_CG_{m2}R_2}}                                                                                                                                                      \\
			\omega_{P2}=\frac{G_{m2}C_C}{C_1C_2+C_C\(C_1+C_2\)}                                                    & \omega_t=\(G_{m1}R_1G_{m2}R_2\)\omega_{P1}                                                                                                                    & \text{Επιλογή }C_C\text{ ώστε }\omega_t<\omega_Z<\omega_{P2}                                            \\
			\mathrm{PSRR}=g_{mN}\(r_{oP}\parallel r_{oN}\)                                                         & \mathrm{SR}\equiv\left.\der{V_{\mathrm{out}}}{t}\right\vert_{\max}=\frac{I_{SS}}{C_L}                                                                                  &                                                                                                         \\
			\multicolumn{3}{|C|}{\mathrm{GB}=A_v(0)\cdot|p_1|=\(g_{m1}g_{m2}R_IR_{II}\)\cdot\(\frac{1}{g_{m2}R_IR_{II}C_C}\)=\frac{g_{m1}}{C_C}}                                                                                                                                                                                                                                             \\
			\multicolumn{3}{|C|}{\mathrm{PM}=\mathrm{Arg}\left[AB\right]=\pm180{\degree}-\arctan{\(\frac{\omega}{|p_1|}\)}-\arctan{\(\frac{\omega}{|p_2|}\)}-\arctan{\(\frac{\omega}{Z}\)}}                                                                                                                                                                                                  \\
			\multicolumn{3}{|l|}{Για περιθώριο φάσης (PM) $45\degree$ και $Z\geqslant 10\cdot\mathrm{GB}$: $|p_2|\geqslant 1.22\cdot\mathrm{GB}$}                                                                                                                                                                                                                                            \\
			\multicolumn{3}{|l|}{Για περιθώριο φάσης (PM) $60\degree$ και $Z\geqslant 10\cdot\mathrm{GB}$: $|p_2|\geqslant 2.22\cdot\mathrm{GB}$}                                                                                                                                                                                                                                            \\
			\multicolumn{3}{|C|}{Z=\frac{1}{C_C\(\displaystyle{\frac{1}{g_{m2}}}-R_Z\)}}                                                                                                                                                                                                                                                                                                     \\[18pt]
			\hline
		\end{tabular}
	\end{center}
\end{table}